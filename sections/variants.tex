\section{Game Variants}\label{sec:variants}
This section outlines a number of gameplay variants players can add into the game should they see fit.
\subsection{Domino Runs}
Players are allowed to make standard domino runs, where the pieces line up end-to-end as a valid way of creating a winning hand. With the requirement that all 13 pieces (11 for 2 players) in your hand must be used with none left over.

Whether you allow for branching on pairs is up to you.

\aside It was left out of the base rules because it was deemed too easy to do.

\subsection{No Milling}
In the No Milling variant, you are not allowed to simply pick up an already revealed piece in a column and put it back at the end.

This is similar to how you can't just pick up and discard the same face-up card in a standard game of Rummy.

\aside Milling was left in because it adds strategic value to the game, being able to effectively ``pass'' a turn or block another player by simply moving a piece from one end to another.

\subsection{Never Close on Pickup}
If you wish to perhaps have a longer and more challenging game, you can disallow closing on pickup, requiring that you always have exactly 13 dominoes (11 for 2 players) at the end of the game.

Should the 12th/14th domino win you the game, it's tough luck, and you have to discard something and reformulate your strategy.

\subsection{Jump-Ins (3-4 players only)}
With Jump-Ins a player out of sequence may call ``Mine!'' when another player discards a piece. Doing so, causes that player to jump in, skipping everyone else.
Play then continues from the caller as if it were their turn.

\subsubsection{...But Only on Completion}\label{sec:jump-in}
Borrowing from Mahjong, Jump-ins can be restricted to only be allowed when they complete a meld. 
Doing so then forces that player to reveal said meld, playing it face-up in front of them. This can clue the other players into the jumper's strategy.

\subsection{Partners}
In a four-player game, you can choose to play 2 on 2 where you are partnered with the person across the table.
This completely changes the dynamic of discarding and scoring, where you might \textit{want} to discard in a way that benefits your partner. May work well with Jump-Ins.

\subsection{Everything Scores the Same}
If you find the scoring rules too complicated, feel free to just treat every type of meld as a run, giving you 1 point for 3-4 tiles and additional points for each extra tile beyond that.

\subsection{No Stealing}
If you find capping and stealing to be unfair, you can leave this rule out.

\subsection{No Royal Melds}
If you think the completion bonus rule is too complicated or are teaching a new player the game and don't want to bog them down with unnecessary complexities, feel free to leave this rule out.
The game is perfectly playable without.

\subsection{Only Open/Closed Melds}
In the basic version of the game, people are allowed to reveal their melds as they see fit.
If you instead wish to agree ahead of time that all melds should be open or closed, then you may also choose to do so.

\subsection{Closed Hand Bonus}\label{sec:closed-hand}
Add a 4 point bonus to the player if they finish without ever revealing their hand.
Pair this with the Jump-Ins on Completion variant (Section~\ref{sec:jump-in}) to make sure their hand opens if they jump in.

\subsection{Conveyor Mahjong}
In Mahjong the aim of the game is to create four sets and a pair.

\paragraph{Rules Changes}

\begin{itemize}
    \item Melds can never be longer than 3 tiles.
    \item For two players, you play to 4 melds (12 tiles).
    \item For 3--4 players, you play to 4 melds and a pair (14 tiles). The pair must be two tiles that share a sum.
    \item Completion bonuses are awarded only if all of the melds in total make up the completed set. 
    \item A 13 tile completed set may ignore the pair requirement.
\end{itemize}

Combine this variant with the Closed Hand Bonus (Section~\ref{sec:closed-hand}) and the Jump-In on Completion (Section~\ref{sec:jump-in}) variants for that extra Mahjongy feel.