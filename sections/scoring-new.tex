\section{Scoring}\label{sec:scoring}
When a player closes their hand by completing their final meld, the round ends and the scoring can begin, the player who successfully closed their hand is henceforth known as the \textit{victor}.

Melds of length 3 or 4 score the victor the meld's value. Any additional tile beyond that awards that meld's value again.

The different types of melds are valued by the difficulty at which they are acquired.

\begin{center}
    \begin{tabular}{c||c|c|c}
              & Runs & Sums & Pairs\\\hline\hline
        Value & 1 & 2 & 3\\
    \end{tabular}
\end{center}

\paragraph{Example Melds}
The table below shows the score that different lengths of melds will net you according to the rules described above.
\begin{center}
    \begin{tabular}{c||c|c|c}
        Length & Runs & Sums & Pairs\\\hline\hline
        3--4 & 1 &  2 &  3 \\
           5 & 2 &  4 &  6 \\
           6 & 3 &  6 &  9 \\
           7 & 4 &  8 & 12 \\
           8 & 5 &  - & 15 \\
           9 & 6 &  - & 18 \\
          10 & 7 &  - & 21 \\
          11 & 8 &  - & 24 \\
          12 & 9 &  - & 27 \\
          13 &10 &  - & 30 \\\hline
          Compl. Bonus & \multicolumn{3}{c}{+10 points}
    \end{tabular}
\end{center}
In addition, the victor gets one point for each unmelded tile held by the opponents.

\note Only the victor scores, not the other players. The other players may try to mitigate the score by shedding and stealing (see below).

\aside You may have noticed that melds of length 3 and 4 score the same. This is there to incentivise players to discard their unmelded tiles onto melds of length 3 without increasing the victor's overall score. It may even reduce it.

\subsection{Shedding}
To potentially reduce the amount of points held by the victor, the other players have the option of \textit{shedding} some of their loose tiles by melding them onto the other players' existing melds.

\subsection{Completion Bonuses (Royal Melds)}
As a juicy reward for finishing the round with a meld containing every possible tile that makes up this meld (a Royal Meld), you are awarded an additional 10 points.

This is particularly useful when dealing with sums---which can never be longer than 7---and can even be as short as just three tiles!

\aside We wanted to reward players who managed to get really hard-to-get hands, the really short 3-tile sums in particular. Having these be scored the same as everything else would dis-incentivise players from pursuing them, but giving 10 bonus points might just make it worth it. The 30 point \textit{13 pairs} hand becomes a total of 40 points instead!

\subsubsection{2 Player Royal Melds}
Due to a small discrepency in the hand-size in a 2-player game (11 tiles), and the maximum possible size of a Royal meld, a player attempting to get all 10 tiles in a suit will inevitably be left over with one tile that cannot be melded into the rest of the hand.

In order to prevent being stuck with one left-over tile that cannot be melded, should you choose to play with completion bonuses, you may choose to ignore this last tile and close the hand anyway. The left-over tile is not scored.

\note This left-over tile rule only ever applies to Royal Melds of length 10 and no other hands.