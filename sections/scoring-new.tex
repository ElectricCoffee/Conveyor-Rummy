\section{Scoring}\label{sec:scoring}
When a player closes their hand by completing their final meld, they reveal it for scrutiny. 
If the hand has been deemed valid by the other players, the game will end immediately, and the scoring begins.

Melds of length 3 or 4 score the victor the meld's value. Any additional tile beyond that awards that meld's value again.

\begin{center}
    \begin{tabular}{c|c}
        Meld & Value\\\hline\hline
        Runs & 1\\
        Sums & 2\\
        Pairs & 3\\
    \end{tabular}
\end{center}

\subsubsection*{Example Melds}
\begin{center}
    \begin{tabular}{r|c|c|c}
        Length & Runs & Sums & Pairs\\\hline\hline
        3--4 & 1 &  2 &  3 \\
           5 & 2 &  4 &  6 \\
           6 & 3 &  6 &  9 \\
           7 & 4 &  8 & 12 \\
           8 & 5 & 10 & 15 \\
    \end{tabular}
\end{center}

In addition, the victor gets one point for each unmelded tile held by the opponents.

\aside You may have noticed that melds of length 3 and 4 score the same. This is there to incentivise players to discard their unmelded tiles onto melds of length 3 without increasing the victor's overall score. It may even reduce it.

\subsection{Completion Bonuses (Royal Melds)}
Finishing the round with a meld containing every possible tile that makes up this meld (a Royal Meld) awards you an additional 10 points.

This is particularly useful when dealing with sums, as some Royal Sums can be as short as just three tiles!

\subsection{Capping and Stealing}